%% See http://www.oxfordjournals.org/our_journals/bioinformatics/for_authors/general.html for author information

\documentclass{bioinfo}
\copyrightyear{2011}
\pubyear{2011}

\begin{document}
\firstpage{1}

\title[KEGGtranslator]{KEGGtranslator: visualizing and translating the KEGG pathway database} %% TODO: Irgendwas mit modelling oder simulation oder so klingt besser.
\author[Wrzodek \textit{et~al}]{Clemens Wrzodek\,$^{1,*}$, Andreas Dr\"ager\,$^{1}$ and Andreas Zell\,$^1$\footnote{to whom correspondence should be addressed}}
\address{$^{1}$Center for Bioinformatics T\"ubingen (ZBIT), University of T\"ubingen, Sand 1, 72076 T\"ubingen, Germany}

\history{Received on XXXXX; revised on XXXXX; accepted on XXXXX}

\editor{Associate Editor: XXXXXXX}

\maketitle

\begin{abstract}

\section{Summary:}
The KEGG PATHWAY database is today a widely used service for pathway based information. It contains manually drawn pathway maps with information about the genes, reactions and relations contained within a pathway. To model these pathways, KEGG introduced an own XML-format and parsers and translators are needed to process the pathway maps for usage in other applications and algorithms.

TODO: Eventuell noch irgendein Nebensatz, aus dem hervorgeht, dass Reaktionen nicht komplett sein m�ssen, sondern nur f�r visualisierung.


We have developed KEGGtranslator, which is an easy-to-use stand-alone application that can translate KGML formatted XML-files to multiple output formats. Unlike other translators, KEGGtranslator supports a plethora of output formats, is able to complete the information in translated documents (i.e. MIRIAM annotations) beyond the scope of the XML-document and amends missing components to fragmentary reactions within the pathway to allow simulations on those.

% [is able to complete [...] beyond the scope of the XML-document] by retrieving additional pathway information from the internet.

%Es gibt ein paar andere translator, aber...
%XML-grafisch ok, f�r simmulationen nicht so toll.
%Translator nutzt KegggAdaptor, kann RkT vervollst�ndigung, ist Stand-Alone-Application

\section{Availability:}
KEGGtranslator is freely available as a webstart application and for download at http://TODO-MAKE-HOMEPAGE. KGML files can be downloaded within the application of manually from \url{ftp://ftp.genome.jp/pub/kegg/xml/kgml}.


\section{Contact:} \href{Clemens.Wrzodek@uni-tuebingen.de}{Clemens.Wrzodek@uni-tuebingen.de}
\end{abstract}

%% TODO: Availability and Implementation section (Java, runs on most OS, blablabla)

\section{Introduction}

Many academic researchers who want to use pathway based information are using the KEGG Pathway database.

TODO: Mehr Infos zu der DB. und mit dem KGML format und XML dateien �berleiten.

However, the content of these KGML-formatted xml-files is limited. Gene names are often not very readable and elements are not well annotated. The reason for this is that KEGG releases those xml-files primary for graphical visualizations of the pathway. For creating functional models of the pathway, the content of the xml-file is not sufficient.

KEGGtranslator completes the translated content of the XML-file with live-annotation of all genes and reactions. Minor deficiencies are corrected (i.e. the name of a gene), new information is being added (i.e. multiple MIRIAM identifier for each gene and reaction or SBO terms describing the function) and some crucial deficiencies are addressed:

KEGG uses colors in the visualization of a pathway to annotate organism specific orthologous genes. Nodes in green represent biological entities that occur in the current organism. Nodes in white represent biological entities, that correspond to genes that are occurring in this pathway in other species, but not in the current one. Translating all those nodes into function models, without caring for the node color would lead to a model, containing invalid genes in the pathway.

Another major deficiency are the reactions. The XML-files only contain those parts of the reaction, that are needed for the graphical representation of the pathway. But they do not always contain the complete chemical equation. KEGGtranslator is able to lookup each reaction and amend the missing components to reactions. This leads to more complete and function correct pathway models.


%Beispiel aus Glycloyse (000010)
%XML file contains:
%===================
%    <reaction name="rn:R05198" type="reversible">
%        <substrate name="cpd:C00469"/>
%        <product name="cpd:C00084"/>
%    </reaction>
%
%Actual reaction is:
%http://www.genome.jp/dbget-bin/www_bget?rn:R05198
%===================
%Ethanol + 2 Cytochrome c <=> 2 Ferrocytochrome c + Acetaldehyde
%C00469 + 2 C00524 <=> 2 C00126 + C00084


TODO: bild um das mit white/green nodes zu zeigen und bild um das mit reaktions zu zeigen

To our knowledge, these major deficiencies are not being corrected by other KEGG converters.


\section{Translation of KGML-files}

In the first step of a translation, KEGGtranslator reads the XML-file and puts all contained elements into an internal data structure. In addition to the XML-file, the KEGG database is queries via the KEGG API for each id in the document (pathway id, entries, recations, relations, substrates, products, etc.). This completes the sparse XML-document with comprehensive information. For example, multiple synonyms and miriam urns of many external databases (Ensembl, EntrezGene, UniProt, ChEBI, Gene Ontology, DrugBank, PDBeChem, and many more) are being assigned to genes.

TODO: Preprocessing (remove white nodes, reaciton complete, etc.) Auch erwaehnen, dass alle optionen ausschaltbar sind. 

After these preprocessing steps, KEGGtranslator branches between two different conversion modes for the actual translation: a functional translation (SBML) and a graphical translation (e.g. GraphML, GML). Depending on the chosen output format, KEGGtranslator determines the way to translate the KGML document.

The functional translation is performed by converting the KGML code to a jSBML data structure. The focus here is to generate valid and specification-conform SBML code that eases e.g. a dynamic simulation of the pathway. Each entry (pathway references, genes, compounds, enzymes, reactions, reaction-modifiers, etc.) is being assigned multiple MIRIAM-URNs and an SBO-Term which describes best the function of the element. Addirionaly, notes are being assigned to each element with human-readable names and synonyms, a description of the element and links to pictures and further information. The user may also choose to add graphical information by putting CellDesigner annotations to the model. The key difference in this mode, to the graphical translation is, that the focus resides on reactions in KGML documents. Besides the already mentioned completion of reactions, each enzymatic modifier is correctly assigned to the reaction by building so called modifierSpeciesReferences. The reversibility of the reaction is annotated and a picture of the reaction equation is being put into the notes.

We have integrated the SBML2LaTeX tool into KEGGtranslator, which allows users to automatically generate a LaTeX or PDF-report to document the SBML-code of the translated pathway.

%
%Functional:
%CellDesigner Annots
%Valid and unique SBML-code (ids, initial assignments, etc.)
%MIRIAM URNs
%SBO-Terms (closest guess)
%Notes (Including pathway picture and links)
%REACTIONS (with modififerSpeciesReferences, reversibel<-> irreversibel


In graphical translations, results can be saved as GraphML, GML or YGF and finally as images of type JPG, GIF, or TGF. In this mode, the KGML data structure is being converted to a yFiles data structure. The focus here lies on the visualization of the pathway. Relations are being translated by inserting arrows with the appropriate style. The styles are directly read from the KGML document. For example, dashed arrows without heads represent bindings or associations and a dotted arrow with a simply head illustrates an indirect effect. Please see the KGML specification for a complete list. As in the function translation, GraphML allows to define custom annotation elements. KEGGtranslator makes use of those, by putting several identifiers (e.g. EntrezGene or Ensembl) and descriptions to the single nodes. From the KGML-document, the shape of the node is translated as well as the colors and labels. Links to descriptive HTML-pages are being setup and hierarchical group nodes are being created for defined compounds. All these features lead to a nice graphical representation of the pathway that provides as many information about the elements as possible.







\begin{methods}
\section{Methods}

Text Text Text Text Text Text  Text Text Text Text Text Text Text Text Text  Text Text Text Text Text Text. Figure \ref{fig:02} shows that the above method  Text Text Text Text  Text Text Text Text Text Text  Text Text.  \citealp{Boffelli03} might want to know about  text text text text
Text Text Text Text Text Text  Text Text Text Text Text Text Text Text Text  Text Text Text Text Text Text. Figure \ref{fig:02} shows that the above method  Text Text Text Text  Text Text Text Text Text Text  Text Text.  \citealp{Boffelli03} might want to know about  text text text text
Text Text Text Text Text Text  Text Text Text Text Text Text Text Text Text  Text Text Text Text Text Text. Figure \ref{fig:02} shows that the above method  Text Text Text Text  Text Text Text Text Text Text  Text Text.  \citealp{Boffelli03} might want to know about  text text text text

\begin{itemize}
\item for bulleted list, use itemize
\item for bulleted list, use itemize
\item for bulleted list, use itemize
\end{itemize}



Text Text Text Text Text Text  Text Text Text Text Text Text Text Text Text  Text Text Text Text Text Text. Figure \ref{fig:02} shows that the above method  Text Text Text Text  Text Text Text Text Text Text  Text Text.  \citealp{Boffelli03} might want to know about  text text text text
Text Text Text Text Text Text  Text Text Text Text Text Text Text Text Text  Text Text Text Text Text Text. Figure \ref{fig:02} shows that the above method  Text Text Text Text  Text Text Text Text Text Text  Text Text.  \citealp{Boffelli03} might want to know about  text text text text
Text Text Text Text Text Text  Text Text Text Text Text Text Text Text Text  Text Text Text Text Text Text. Figure \ref{fig:02} shows that the above method  Text Text Text Text  Text Text Text Text Text Text  Text Text.  \citealp{Boffelli03} might want to know about  text text text text
Text Text Text Text Text Text  Text Text Text Text Text Text Text Text Text  Text Text Text Text Text Text. Figure \ref{fig:02} shows that the above method  Text Text Text Text  Text Text Text Text Text Text  Text Text.  \citealp{Boffelli03} might want to know about  text text text text
Text Text Text Text Text Text  Text Text Text Text Text Text Text Text Text  Text Text Text Text Text Text.


Text Text Text Text Text Text  Text Text Text Text Text Text Text Text Text  Text Text Text Text Text Text. Figure \ref{fig:02} shows that the above method  Text Text Text Text  Text Text Text Text Text Text  Text Text.  \citealp{Boffelli03} might want to know about  text text text text
Text Text Text Text Text Text  Text Text Text Text Text Text Text Text Text  Text Text Text Text Text Text. Figure \ref{fig:02} shows that the above method  Text Text Text Text  Text Text Text Text Text Text  Text Text.  \citealp{Boffelli03} might want to know about  text text text text
Text Text Text Text Text Text  Text Text Text Text Text Text Text Text Text  Text Text Text Text Text Text. Figure \ref{fig:02} shows that the above method  Text Text Text Text  Text Text Text Text Text Text  Text Text.  \citealp{Boffelli03} might want to know about  text text text text



Text Text Text Text Text Text  Text Text Text Text Text Text Text Text Text  Text Text Text Text Text Text. Figure \ref{fig:02} shows that the above method  Text Text Text Text  Text Text Text Text Text Text  Text Text.  \citealp{Boffelli03} might want to know about  text text text text
Text Text Text Text Text Text  Text Text Text Text Text Text Text Text Text  Text Text Text Text Text Text. Figure \ref{fig:02} shows that the above method  Text Text Text Text  Text Text Text Text Text Text  Text Text.  \citealp{Boffelli03} might want to know about  text text text text
Text Text Text Text Text Text  Text Text Text Text Text Text Text Text Text  Text Text Text Text Text Text. Figure \ref{fig:02} shows that the above method  Text Text Text Text  Text Text Text Text Text Text  Text Text.  \citealp{Boffelli03} might want to know about  text text text text


Text Text Text Text Text Text  Text Text Text Text Text Text Text Text Text  Text Text Text Text Text Text. Figure \ref{fig:02} shows that the above method  Text Text Text Text  Text Text Text Text Text Text  Text Text.  \citealp{Boffelli03} might want to know about  text text text text
Text Text Text Text Text Text  Text Text Text Text Text Text Text Text Text  Text Text Text Text Text Text. Figure \ref{fig:02} shows that the above method  Text Text Text Text  Text Text Text Text Text Text  Text Text.  \citealp{Boffelli03} might want to know about  text text text text
Text Text Text Text Text Text  Text Text Text Text Text Text Text Text Text  Text Text Text Text Text Text. Figure \ref{fig:02} shows that the above method  Text Text Text Text  Text Text Text Text Text Text  Text Text.  \citealp{Boffelli03} might want to know about  text text text text



\begin{table}[!t]
\processtable{This is table caption\label{Tab:01}}
{\begin{tabular}{llll}\toprule
head1 & head2 & head3 & head4\\\midrule
row1 & row1 & row1 & row1\\
row2 & row2 & row2 & row2\\
row3 & row3 & row3 & row3\\
row4 & row4 & row4 & row4\\\botrule
\end{tabular}}{This is a footnote}
\end{table}

\end{methods}

\begin{figure}[!tpb]%figure1
%\centerline{\includegraphics{fig01.eps}}
\caption{Caption, caption.}\label{fig:01}
\end{figure}

\begin{figure}[!tpb]%figure2
%\centerline{\includegraphics{fig02.eps}}
\caption{Caption, caption.}\label{fig:02}
\end{figure}

\section{Discussion}

Text Text Text Text Text Text  Text Text Text Text Text Text Text Text Text  Text Text Text Text Text Text. Figure \ref{fig:02} shows that the above method  Text Text Text Text  Text Text Text Text Text Text  Text Text.  \citealp{Boffelli03} might want to know about  text text text text
Text Text Text Text Text Text  Text Text Text Text Text Text Text Text Text  Text Text Text Text Text Text. Figure \ref{fig:02} shows that the above method  Text Text Text Text  Text Text Text Text Text Text  Text Text.  \citealp{Boffelli03} might want to know about  text text text text
Text Text Text Text Text Text  Text Text Text Text.




Table~\ref{Tab:01} shows that Text Text Text Text Text  Text Text Text Text Text Text. Figure \ref{fig:02} shows that
the above method Text Text. Text Text Text  Text Text Text Text Text Text. Figure \ref{fig:02} shows that
the above method Text Text. Text Text Text  Text Text Text Text Text Text. Figure \ref{fig:02} shows that
the above method Text Text.









%%%%%%%%%%%%%%%%%%%%%%%%%%%%%%%%%%%%%%%%%%%%%%%%%%%%%%%%%%%%%%%%%%%%%%%%%%%%%%%%%%%%%
%
%     please remove the " % " symbol from \centerline{\includegraphics{fig01.eps}}
%     as it may ignore the figures.
%
%%%%%%%%%%%%%%%%%%%%%%%%%%%%%%%%%%%%%%%%%%%%%%%%%%%%%%%%%%%%%%%%%%%%%%%%%%%%%%%%%%%%%%






\section{Conclusion}

(Table~\ref{Tab:01}) Text Text Text Text Text Text  Text Text Text Text Text Text Text Text Text  Text Text Text Text Text Text. Figure \ref{fig:02} shows that the above method  Text Text Text Text  Text Text Text Text Text Text  Text Text.  \citealp{Boffelli03} might want to know about  text text text text
Text Text Text Text Text Text  Text Text Text Text Text Text Text Text Text  Text Text Text Text Text Text. Figure \ref{fig:02} shows that the above method  Text Text Text Text  Text Text Text Text Text Text  Text Text.  \citealp{Boffelli03} might want to know about  text text text text
Text Text Text Text Text Text  Text Text Text Text Text Text Text Text Text  Text Text Text Text Text Text. Figure \ref{fig:02} shows that the above method  Text Text Text Text  Text Text Text Text Text Text  Text Text.



Text Text Text Text Text Text  Text Text Text Text Text Text Text Text Text  Text Text Text Text Text Text. Figure \ref{fig:02} shows that the above method  Text Text Text Text  Text Text Text Text Text Text  Text Text.  \citealp{Boffelli03} might want to know about  text text text text





\begin{enumerate}
\item this is item, use enumerate
\item this is item, use enumerate
\item this is item, use enumerate
\end{enumerate}

Text Text Text Text Text Text  Text Text Text Text Text Text Text Text Text  Text Text Text Text Text Text. Figure \ref{fig:02} shows that the above method  Text Text Text Text  Text Text Text Text Text Text  Text Text.  \citealp{Boffelli03} might want to know about  text text text text
Text Text Text Text Text Text  Text Text Text Text Text Text Text Text Text  Text Text Text Text Text Text. Figure \ref{fig:02} shows that the above method  Text Text Text Text  Text Text Text Text Text Text  Text Text.  \citealp{Boffelli03} might want to know about  text text text text
Text Text Text Text Text Text  Text Text Text Text Text Text Text Text Text  Text Text Text Text Text Text.






Text Text Text Text Text Text  Text Text Text Text Text Text Text Text Text  Text Text Text Text Text Text. Figure \ref{fig:02} shows that the above method  Text Text Text Text


\section*{Acknowledgement}
Text Text Text Text Text Text  Text Text.  \citealp{Boffelli03} might want to know about  text text text text

\paragraph{Funding\textcolon} Text Text Text Text Text Text  Text Text.

%\bibliographystyle{natbib}
%\bibliographystyle{achemnat}
%\bibliographystyle{plainnat}
%\bibliographystyle{abbrv}
%\bibliographystyle{bioinformatics}
%
%\bibliographystyle{plain}
%
%\bibliography{Document}


\begin{thebibliography}{}
\bibitem[Bofelli {\it et~al}., 2000]{Boffelli03} Bofelli,F., Name2, Name3 (2003) Article title, {\it Journal Name}, {\bf 199}, 133-154.

\bibitem[Bag {\it et~al}., 2001]{Bag01} Bag,M., Name2, Name3 (2001) Article title, {\it Journal Name}, {\bf 99}, 33-54.

\bibitem[Yoo \textit{et~al}., 2003]{Yoo03}
Yoo,M.S. \textit{et~al}. (2003) Oxidative stress regulated genes
in nigral dopaminergic neurnol cell: correlation with the known
pathology in Parkinson's disease. \textit{Brain Res. Mol. Brain
Res.}, \textbf{110}(Suppl. 1), 76--84.

\bibitem[Lehmann, 1986]{Leh86}
Lehmann,E.L. (1986) Chapter title. \textit{Book Title}. Vol.~1, 2nd edn. Springer-Verlag, New York.

\bibitem[Crenshaw and Jones, 2003]{Cre03}
Crenshaw, B.,III, and Jones, W.B.,Jr (2003) The future of clinical
cancer management: one tumor, one chip. \textit{Bioinformatics},
doi:10.1093/bioinformatics/btn000.

\bibitem[Auhtor \textit{et~al}. (2000)]{Aut00}
Auhtor,A.B. \textit{et~al}. (2000) Chapter title. In Smith, A.C.
(ed.), \textit{Book Title}, 2nd edn. Publisher, Location, Vol. 1, pp.
???--???.

\bibitem[Bardet, 1920]{Bar20}
Bardet, G. (1920) Sur un syndrome d'obesite infantile avec
polydactylie et retinite pigmentaire (contribution a l'etude des
formes cliniques de l'obesite hypophysaire). PhD Thesis, name of
institution, Paris, France.

\end{thebibliography}
\end{document}
